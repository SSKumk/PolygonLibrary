\documentclass[a4paper,12pt]{article}

\usepackage{cmap}                   % поиск в PDF
\usepackage[T2A]{fontenc}           % кодировка
\usepackage[utf8]{inputenc}         % кодировка исходного текста
\usepackage[english,russian]{babel} % локализация и переносы
\usepackage[a4paper, margin=20mm]{geometry}
\usepackage{hyperref}               % активные сслыки

% Packages
\usepackage{amsmath,amsfonts,amssymb,amsthm,mathtools} % AMS
\usepackage{mathrsfs} % \mathscr{} ажурный шрифт

\usepackage{indentfirst} %абзацный отступ
\usepackage{ifthen}      %поддержка \ifthanelse{}{}{}
\usepackage{graphicx}
\usepackage{verbatim}
\usepackage[shortcuts]{extdash}




\newcommand{\R}[1]{% R - евклидово пространство
  \ifthenelse{\equal{#1}{}}%
  {\mathbb{R}}%
  {\mathbb{R}^{#1}}}


\newcommand{\scalprod}[3][]{#1\langle #2, #3 #1\rangle} % скалярное происзведение


\DeclareMathOperator{\proj}{\texttt{proj}} % проекция


\begin{document}

  \textbf{Алгоритм выработки управления на очередной шаг}


  \begin{equation}
    \label{eq:OrigGame}
    \begin{aligned}
      & \dot z = A(t)z + B(t)u + C(t)v,\\
      & t \in [t_0, T], \quad \quad z \in \R{n},\\
      & u \in \mathbf P \subset \R{p}, \quad v \in \mathbf Q \subset \R{q}, \\
      & z(T) \in \mathbf M + \mathbf M^\bot.
    \end{aligned}
  \end{equation}

\begin{equation}
  \label{eq:EqGame}
  \begin{aligned}
    & \dot x = D(t)u + E(t)v, \\
    & t \in [t_0, T], \  x \in \R{d}, \  u \in \mathbf P, \  v \in \mathbf Q, \\
    & D(t) = X_{s_1,s_2,\ldots,s_d}(T,t)B(t),  \\
    & E(t) = X_{s_1,s_2,\ldots,s_d}(T,t)C(t),   \\
    & x(T) \in \mathbf M.
  \end{aligned}
\end{equation}


$\mathcal{P}_i = (-\Delta)D(t_i)P$, $\mathcal{Q}_i = -\Delta E(t_i)Q$


  Исходная фазовая переменная $z \in \R{n}$, эквивалентная $x \in \R{d}$, $1 \leqslant \dim x \leqslant \dim z$.

  Сечения максимального стабильного моста в моменты времени $t_i$: $W_i \subset \R{d}$.

  \bigskip

  Дано: $x \in \R{n}$

  Найти: $u \in P$
  \begin{enumerate}
    \item Найти ближайшую точку $h \in W_i$ к $x$;
    \item В $\mathcal{P}_i$ найти точку $p$, такую, что $\scalprod{h-x}{p}$ максимальное;
    \item Найти любую точку $u \in P$, такую что $\proj(u, \mathcal{P}_i) = p$;
  \end{enumerate}
\end{document}