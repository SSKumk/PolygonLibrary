% Preamble
\documentclass[a4paper,12pt]{article}

\usepackage{cmap}                   % поиск в PDF
\usepackage[T2A]{fontenc}           % кодировка
\usepackage[utf8]{inputenc}         % кодировка исходного текста
\usepackage[english,russian]{babel} % локализация и переносы
\usepackage[a4paper, margin=20mm]{geometry}
\usepackage{hyperref}               % активные сслыки

% Packages
\usepackage{amsmath,amsfonts,amssymb,amsthm,mathtools} % AMS
\usepackage{mathrsfs} % \mathscr{} ажурный шрифт

\usepackage{indentfirst} %абзацный отступ
\usepackage{ifthen}      %поддержка \ifthanelse{}{}{}
\usepackage{graphicx}
\usepackage{verbatim}
\usepackage[shortcuts]{extdash}
%-----------
%\newcommand{\}{}
\newcommand{\scalprod}[3][]{#1\langle #2, #3 #1\rangle} % Скалярное произведение
\DeclareMathOperator{\argmin}{argmin}
\DeclareMathOperator{\argmax}{argmax}


\renewcommand{\.}{\hspace{0.2ex}}


\begin{document}

\section{Алгоритм перевода гиперплоскостного описания многогранника в вершинное представление}

  Пусть дан набор гиперплоскостей, описывающий многогранник. Нормали смотрят наружу.

  \textit{Этап 1}. Найти какую-нибудь вершину многогранника. \\
  Наивный перебор / симплекс-метод.

  \textit{Этап 2}. Найти оставшиеся вершины многогранника.

  Создадим очередь вершин из которых мы ещё не строили рёбра. Инициализируем её вершиной найденной в первом этапе.

  Пока очередь не пуста:

  Берём очередную вершину $z$ из очереди. Найдём набор $\mathcal{H}_z$ гиперплоскостей проходящих через эту точку.

  Перебираем всевозможные наборы из $d-1$ гиперплоскости.

  Для каждого такого набора с помощью метода Г-Ш построим направляющий вектор~$v$. Если для любого вектора нормали $n$ из набора $\mathcal{H}_z$ $\scalprod{v}{n} \leqslant 0$, то такой вектор задаёт прямую содержащую ребро, обозначим его $v_*$.

  Найдём первую точку пересечения вектора $v_*$ с гиперплоскостями из набора $\mathcal{H}_z$. Однако вектор $v_*$ может <<смотреть>> наружу многогранника, поэтому заодно найдём и пересечение в другую сторону.
  $$
  t_i \leftarrow A_i^{-1} \frac{b_i - A_i z}{v_* + z}; \quad j = \argmin_i(t_i \. | \. t_i > 0); \quad k = \argmax_i(t_i \. | \. t_i < 0)
  $$
  Возможно, что одного из пересечений не будет, эту ситуацию нужно обработать отдельно.

  Далее, подставляем найденные индексы в уравнение прямой и находим две точки. После чего обе подставляем в каждую гиперплоскость из $\mathcal{H}_z$ и выбираем ту точку $p_*$, которая удовлетворяет всем неравенствам.

  Запоминаем $p_*$ как вершину многогранника и добавляем её в очередь.
\end{document}