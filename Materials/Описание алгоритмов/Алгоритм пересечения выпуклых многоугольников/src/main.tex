% Preamble
\documentclass[a4paper,12pt]{article}

\usepackage{cmap}                   % поиск в PDF
\usepackage[T2A]{fontenc}           % кодировка
\usepackage[utf8]{inputenc}         % кодировка исходного текста
\usepackage[english,russian]{babel} % локализация и переносы
\usepackage[a4paper, margin=20mm]{geometry}

% Packages
\usepackage{amsmath,amsfonts,amssymb,amsthm,mathtools} % AMS

\usepackage{algorithm}
\usepackage{algpseudocode}
\usepackage{graphicx}
\usepackage{algorithmicx}
\usepackage{verbatim}
\usepackage{indentfirst}


% Добавляем свои блоки
\makeatletter
\algblock[ALGORITHMBLOCK]{BeginAlgorithm}{EndAlgorithm}
\algblock[BLOCK]{BeginBlock}{EndBlock}
\makeatother

\floatname{algorithm}{Алгоритм}


% Перевод команд псевдокода
\algrenewcommand\algorithmicwhile{\textbf{До тех пока}}
\algrenewcommand\algorithmicdo{\textbf{выполнять}}
\algrenewcommand\algorithmicrepeat{\textbf{повторять}}
\algrenewcommand\algorithmicuntil{\textbf{пока не выполняется:}}
\algrenewcommand\algorithmicend{\textbf{Конец}}
\algrenewcommand\algorithmicif{\textbf{если}}
\algrenewcommand\algorithmicelse{\textbf{иначе}}
\algrenewcommand\algorithmicthen{\textbf{тогда}}
\algrenewcommand\algorithmicfor{\textbf{Цикл}}
\algrenewcommand\algorithmicforall{\textbf{Выполнить для всех}}
\algrenewcommand\algorithmicfunction{\textbf{Функция}}
\algrenewcommand\algorithmicprocedure{\textbf{Процедура}}
\algrenewcommand\algorithmicloop{\textbf{Зациклить}}
\algrenewcommand\algorithmicrequire{\textbf{Условия:}}
\algrenewcommand\algorithmicensure{\textbf{Обеспечивающие условия:}}
\algrenewcommand\algorithmicreturn{\textbf{Возвратить}}
\algrenewtext{EndWhile}{\textbf{Конец цикла}}
\algrenewtext{EndLoop}{\textbf{Конец зацикливания}}
\algrenewtext{EndFor}{\textbf{Конец цикла}}
\algrenewtext{EndFunction}{\textbf{Конец функции}}
\algrenewtext{EndProcedure}{\textbf{Конец процедуры}}
\algrenewtext{EndIf}{\textbf{конец если}} %Конец условия
\algrenewtext{EndFor}{\textbf{Конец цикла}}
\algrenewtext{BeginAlgorithm}{\textbf{Начало алгоритма}}
\algrenewtext{EndAlgorithm}{\textbf{Конец алгоритма}}
\algrenewtext{ElsIf}{\textbf{иначе если }}



\newcommand{\p}{\hat{p}}
\newcommand{\q}{\hat{q}}
\newcommand{\skewp}{\wedge} %skew -- косой, наклонный; (p)roduct
\newcommand{\dotp}[2]{\langle #1, #2 \rangle} %skew -- косой, наклонный; (p)roduct
\newcommand{\hp}[1]{H^+(#1\/)}
\newcommand{\hpc}[1]{\bar{H}^+(#1\/)}

\def\`{\kern 1pt}

% Document
\begin{document}

\section*{Алгоритм пересечения двух выпуклых \\ многоугольников}

Предполагаем, что используется аффинная плоскость, то есть векторы прикладываются не обязательно к началу координат. Для ненулевого вектора $a$ обозначим $(a)$ прямую, имеющую направляющий вектор, параллельный $a$, и проходящую через начало вектора $a$ (или, что то же самое, проходящую через точки начала и конца вектора $a$). Для вектора~$a$ обозначим $[a]$ отрезок, соединяющий начальную и конечную точки~$a$. Для векторов $a$ и $b$ обозначим $\dotp{a}{b}$ их скалярное произведение: $\dotp{a}{b} = a_x b_x + a_y b_y$. Для векторов $a$ и $b$ обозначим $a \skewp b$ их антисимметричное (косое) произведение: $a \skewp b = a_x b_y - a_y b_x$.

Пусть даны два выпуклых многоугольника $P$ и $Q$ с наборами вершин $V(P) = [p_0,p_1,\ldots,p_{n-1}]$ и $V(Q) = [q_0,q_1,\ldots,q_{k-1}]$, заданные цикличными списками своих вершин в порядке обхода против часовой стрелки. Полагаем, что в многоугольниках все внутренние углы \textit{строго} меньше~$180^\circ$.

Пусть $p \in V(P)$, $q \in V(Q)$ произвольные вершины $P$ и $Q$, соответственно. Обозначим $p_-$ точку предшествующую $p$, а $p_+$ следующую за $p$ в циклическом порядке. Вектор из $p_-$ в $p$ обозначим $\p$. Для $q$ аналогично.

$\hp{\p}$~--- открытая левая полуплоскость, задаваемая вектором $\p$\`:
$$
\hp{\p} = \{x \in \mathbb{R}^2 \ | \  \p \skewp (x - p_-) > 0\}.
$$

Если выполняются общие предположения, то правила перемещения векторов описываются следующей таблицей:
$$
\begin{array}{ccc}
    \hline
    \p \skewp \q & \text { Условие полуплоскости } & \text { Двигаем } \\
    \hline
    \geqslant 0  & q \in \hp{\p}                   & p                 \\
    \geqslant 0  & q \notin \hp{\p}                & q                 \\
    \hline
    <0           & p \in \hp{\q}                   & q                 \\
    <0           & p \notin \hp{\q}                & p                 \\
    \hline
\end{array}
\label{table:AdvanceRules}
$$


\begin{algorithm}
    \caption{Алгоритм пересечения двух выпуклых многоугольников}\label{alg:IntersectionTwoConvPolyg}
    \begin{algorithmic}[1]
        \State Вход:  $P$ и $Q$
        \State Выход: $R = P \cap Q$
        \State Выбрать произвольно $p \in V(P)$ и $q \in V(Q)$
        \State $V(R) \gets \varnothing$
        \State \texttt{inside} $\gets$ \texttt{unknown}
        \Repeat
            \State (\texttt{crossType}, \texttt{point}) $\gets$ Пересечь($[\p]$, $[\q]$)
            \If{\texttt{crossType} $\equiv SinglePoint$}
                \If{\texttt{point} совпадает с первой точкой в $V(R)$}
                    \State закончить алгоритм
                \ElsIf
                        {$p \in$ $\hp{\q}$} \textbf{тогда} \texttt{inside} $\gets$ \texttt{inP}
                \ElsIf
                        {$q \in$ $\hp{\p}$} \textbf{тогда} \texttt{inside} $\gets$ \texttt{inQ}
                    \State \Comment иначе \texttt{inside} не меняется.
                \EndIf
                \State \Call{Добавить}{\texttt{point}}
            \EndIf

            \Statex
            \State \texttt{cross} $\gets \text{sign}(\p \skewp \q \,)$
            \If{\texttt{crossType} $\equiv Overlap$ И $\langle \p, \q \, \rangle < 0$}
                \State $V(R) \gets \varnothing$
                \State закончить алгоритм
%            \ElsIf
%                    {\texttt{cross} = 0 И $p \notin \hpc{\q}$ И $q \notin \hpc{\p}$}
%                \Comment Эвристика, ускоряющая обнаружение пустоты пересечения в некоторой ситуации необщего положения; можно обойтись без неё.
%                \State $V(R) \gets \varnothing$
%                \State закончить алгоритм
            \ElsIf
                    {\texttt{crossType} $\equiv Overlap$}
                \State \Call{Двигать}{$p$, ЛОЖЬ}.
            \ElsIf
                    {\texttt{cross} $\geqslant$ 0}
                \If{$q \in \hp{\p}$}
                    \State  \Call{Двигать}{$p$, \texttt{inside} $\equiv$ \texttt{inP}}
                \Else
                    \State \Call{Двигать}{$q$, \texttt{inside} $\equiv$ \texttt{inQ}}
                \EndIf
            \Else
                \Comment {если \texttt{cross} < 0}
                \If{$p \in \hp{\q}$}
                    \State  \Call{Двигать}{$q$, \texttt{inside} $\equiv$ \texttt{inQ}}
                \Else
                    \State \Call{Двигать}{$p$, \texttt{inside} $\equiv$ \texttt{inP}}
                \EndIf
            \EndIf
        \Until{Оба списка пройдены ИЛИ один из них пройден два раза}
        \State Если $V(R) \equiv \varnothing$, то обработать случаи, $P \subset Q$, $Q \subset P$, $R \equiv \varnothing$. Делается за логарифм двоичным поиском по углу.
        \algstore{bkbreak}
    \end{algorithmic}


\end{algorithm}
\begin{algorithm}
    \begin{algorithmic}
        \algrestore{bkbreak}
        \Procedure{Добавить}{Вершина $v$}
            \If{$V(R) \equiv \varnothing$ ИЛИ $v \neq$ последней точке в $V(R)$}
                \State $V(R) \gets v$
            \EndIf
        \EndProcedure

        \Statex
        \Procedure{Двигать}{Вершина $v$, флаг\_Выводить}
            \If{флаг\_Выводить}
                \State \Call{Добавить}{$v$}
            \EndIf
            \State перейти к следующей вершине многоугольника, соответствующего вершине $v$.
        \EndProcedure
    \end{algorithmic}
\end{algorithm}

\end{document}
